\documentclass[12pt,reqno]{amsart}

\usepackage{amsthm,amsmath,amssymb}
\usepackage{mathtools}
\usepackage{proof}
\usepackage{centernot}
\usepackage{xcolor}
\usepackage{graphicx}
\usepackage[T1]{fontenc}
\usepackage{courier}
\usepackage{enumitem}
\usepackage{hyperref}
\usepackage{array}
\usepackage{multirow}
\usepackage{algorithmic}
\usepackage{textcomp}
\usepackage{algorithm}
\usepackage{cite}
\usepackage{listings}
\lstset{basicstyle=\ttfamily\tiny, columns=fullflexible, language=Python, morekeywords={logical_and, log, exp, dot, sqrt, ones, identity}}
\definecolor{mySucces}{RGB}{40, 167, 69}
\definecolor{myFail}{RGB}{220, 53, 69}

\newcommand{\code}[1]{\texttt{#1}}
\newcommand{\st}[0]{\text{ s.t. }}
\newcommand{\where}[0]{\text{ where }}
\newcommand{\mand}[0]{\text{ and }}
\newcommand{\msgspc}[0]{\mathcal{M}}
\newcommand{\cphspc}[0]{\mathcal{C}}
\newcommand{\keyspc}[0]{\mathcal{K}}
\newcommand{\advrs}[0]{\mathcal{A}}
\newcommand{\distin}[0]{\mathcal{D}}
\newcommand{\oracle}[0]{\mathcal{O}}
\newcommand{\correctans}[0]{\colorbox{mySucces}{CORRECT}}
\newcommand{\falseans}[0]{\colorbox{myFail}{FALSE}}
\newcommand\MyBox[2]{
  \fbox{\lower0.75cm
    \vbox to 1.7cm{\vfil
      \hbox to 1.7cm{\hfil\parbox{1.4cm}{#1\\#2}\hfil}
      \vfil}%
  }%
}
\graphicspath{ {./} }
\newtheorem{theorem}{Theorem}[section]
\newtheorem{axiom}[theorem]{Axiom}
\newtheorem{case}[theorem]{Case}
\newtheorem{claim}[theorem]{Claim}
\newtheorem{conclusion}[theorem]{Conclusion}
\newtheorem{condition}[theorem]{Condition}
\newtheorem{conjecture}[theorem]{Conjecture}
\newtheorem{corollary}[theorem]{Corollary}
\newtheorem{criterion}[theorem]{Criterion}
\newtheorem{definition}[theorem]{Definition}
\newtheorem{example}[theorem]{Example}
\newtheorem{exercise}[theorem]{Exercise}
\newtheorem{lemma}[theorem]{Lemma}
\newtheorem{notation}[theorem]{Notation}
\newtheorem{problem}[theorem]{Problem}
\newtheorem{proposition}[theorem]{Proposition}
\newtheorem{remark}[theorem]{Remark}
\newtheorem{solution}[theorem]{Solution}
\newtheorem{summary}[theorem]{Summary}    

\begin{document}

\begin{center}
\large\textbf{Homework 11 \\ COMP543 Fall 2020 - Modern Cryptography \\}
\normalsize\textbf{ Erhan Tezcan 0070881 \\ 19.12.2020} \\
\end{center}

\begin{center}
\line(1,0){250}
\end{center}

%
%\begin{enumerate}[label=\alph*.]
% \item Explain input, output, and the purpose of each algorithm (Key Generation, Encryption, Decryption). 
% \item What are the key space, the message space, and the ciphertext space?
% \item Formally define the   correctness   requirement of an encryption scheme.
% \end{enumerate}
%

%
%\begin{algorithm}
%\caption{\code{DivideRounds} for a new event \code{x}}
%\label{alg:round}
%\begin{algorithmic}
%\STATE $r \gets \text{max}(selfParent.Round, otherParent.Round)$
%\IF {$x \text{ strongly sees } > \frac{2n}{3} \text{ round } r \text{ witnesses}$} 
%        \STATE $x.Round \gets r+1$
%\ELSE
%        \STATE $x.Round \gets r$
%\ENDIF 
%\STATE $x.witness \gets x.Round > x.selfParent.Round$
%\end{algorithmic}
%\end{algorithm}
%

\section{Quesitons}
\textbf{Q1:} A (digital) signature scheme consists of three probabilistic polynomial-time algorithms (Gen, Sign, Vrfy). Explain each algorithm with its input and output.

\textbf{A1:} These 3 algorithms are:
\begin{itemize}
	\item $Gen$: on input a security parameter $1^n$, outputs public key $pk$ and secret key $sk$.
	\item $Sign$: on input a private key $sk$ and a message $m$, outputs a signature $\sigma$.
	\item $Vrfy$: on input a public key $pk$, message $m$ and a signature $\sigma$, outputs 1 or 0.
\end{itemize}

\vspace{20px}
\textbf{Q2:} Compare signature schemes with MAC schemes.

\textbf{A2:} 
For MACs:
\begin{itemize}
	\item Only the holder of the key can verify it.
	\item You can't transfer a MAC to some other party.
	\item Does not have non-repudiation.
	\item Shorter in length and faster to compute.
\end{itemize}

For Digital Signatures:
\begin{itemize}
	\item Anyone can verify it (with the public key).
	\item You can transfer the signature to some other party.
	\item Has non-repudiation.
	\item Longer keys than MAC, slower to compute.
\end{itemize}

\vspace{20px}
\textbf{Q3:} Formally define security (existential unforgeability against adaptive chosen message attacks) of a signature scheme.

\textbf{A3:} The $\code{Sig-Forge}_{\advrs,\Pi}(n)$ experiment is defined in the algorithm 1 below.
\begin{algorithm}
\caption{$\code{Sig-Forge}_{\advrs,\Pi}(n)$ experiment}
\begin{algorithmic}
\STATE \code{Chal} generates a public and private key pair $(pk, sk) \gets Gen(1^n)$.
\STATE $\advrs$ obtains $pk$, and is given oracle access to $Sign(.)$. It can send a message $m'$ and it will have in return $\sigma'=Sign_{sk}(m')$.
\STATE After some time, $\advrs$ send $m, \sigma$ to \code{Chal}, where $m$ has not been used in oracle access before.
\IF{$Vrfy_{pk}(m, \sigma) = 1$}
	\STATE output 1, $\advrs$ wins.
\ELSE
	\STATE output 0.
\ENDIF
\end{algorithmic}
\end{algorithm}

\vspace{20px}
\textbf{Q4:} Let $\Pi$ be a secure signature scheme for messages of fixed length $l$. Construct a signature scheme based on $\Pi$ that is secure for arbitrary-length messages. You can make use of any crypto-primitive you have learned. (No security proof is needed.)

\textbf{A4:} This is known as Hash-and-Sign paradigm. Let $\Pi$ be $(Gen, Sign, Vrfy)$, we will construct $\Pi'$.
\begin{itemize}
	\item $Gen'$: Same as $Gen$.
	\item $Sign$: on input a private key $sk$ and a message $m$, outputs a $Sign_{sk}(H(m))$.
	\item $Vrfy$: on input a public key $pk$, message $m$ and a signature $\sigma$, outputs $Vrfy_{pk}(H(m),\sigma)$.
\end{itemize}
Here $H$ is a collision resistant hash function. 

\vspace{20px}
\textbf{Q5:} Assume Charlie, acting as a certificate authority (CA), issues a certificate for Bob. Charlie’s signature public key is $pk_C$ and he always issues certificates only to the trustworthy people. Assume further that Bob’s key $pk_B$ is a public key for a signature scheme. Charlie’s certificate for Bob looks like $cert_{C\rightarrow B} = Sign_{sk_C}(\text{Bob’s key is }pk_B)$.
Bob issues a certificate for Alice of the form $cert_{B \rightarrow A} = Sign_{sk_B}(\text{Alice’s key is }pk_A)$.
Now, Alice wants to communicate with some fourth party Dave who knows Charlie’s public key $pk_C$ (but not Bob’s). Assume Charlie is offline now. Suggest a solution by which Alice can convince Dave that $pk_A$ is her authentic key, without any party having access to Charlie (since Charlie is offline).

\textbf{A5:} Alice sends $\langle \textrm{Alice}, pk_A, cert_{B \to A} \rangle$ to Dave. Dave can then ask Bob for $cert_{C \to B}$, which Dave can verify with $pk_C$. After that verification, Dave knows that $B$ must be trustworthy because $C$ issued a certificate to it. Furthermore, it can trust $pk_B$ and use this $pk_B$ to verify $cert_{B \to A}$, which reveals that $pk_A$ is the correct public key of Alice.
\end{document}