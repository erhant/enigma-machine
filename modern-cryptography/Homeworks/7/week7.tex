\documentclass[12pt,reqno]{amsart}

\usepackage{amsthm,amsmath,amssymb}
\usepackage{mathtools}
\usepackage{proof}
\usepackage{centernot}
\usepackage{xcolor}
\usepackage{graphicx}
\usepackage[T1]{fontenc}
\usepackage{courier}
\usepackage{enumitem}
\usepackage{hyperref}
\usepackage{array}
\usepackage{multirow}
\usepackage{algorithmic}
\usepackage{textcomp}
\usepackage{algorithm}
\usepackage{cite}
\usepackage{listings}
\lstset{basicstyle=\ttfamily\tiny, columns=fullflexible, language=Python, morekeywords={logical_and, log, exp, dot, sqrt, ones, identity}}
\definecolor{mySucces}{RGB}{40, 167, 69}
\definecolor{myFail}{RGB}{220, 53, 69}

\newcommand{\code}[1]{\texttt{#1}}
\newcommand{\st}[0]{\text{ s.t. }}
\newcommand{\where}[0]{\text{ where }}
\newcommand{\mand}[0]{\text{ and }}
\newcommand{\msgspc}[0]{\mathcal{M}}
\newcommand{\cphspc}[0]{\mathcal{C}}
\newcommand{\keyspc}[0]{\mathcal{K}}
\newcommand{\advrs}[0]{\mathcal{A}}
\newcommand{\distin}[0]{\mathcal{D}}
\newcommand{\oracle}[0]{\mathcal{O}}
\newcommand{\correctans}[0]{\colorbox{mySucces}{CORRECT}}
\newcommand{\falseans}[0]{\colorbox{myFail}{FALSE}}
\newcommand\MyBox[2]{
  \fbox{\lower0.75cm
    \vbox to 1.7cm{\vfil
      \hbox to 1.7cm{\hfil\parbox{1.4cm}{#1\\#2}\hfil}
      \vfil}%
  }%
}
\graphicspath{ {./} }
\newtheorem{theorem}{Theorem}[section]
\newtheorem{axiom}[theorem]{Axiom}
\newtheorem{case}[theorem]{Case}
\newtheorem{claim}[theorem]{Claim}
\newtheorem{conclusion}[theorem]{Conclusion}
\newtheorem{condition}[theorem]{Condition}
\newtheorem{conjecture}[theorem]{Conjecture}
\newtheorem{corollary}[theorem]{Corollary}
\newtheorem{criterion}[theorem]{Criterion}
\newtheorem{definition}[theorem]{Definition}
\newtheorem{example}[theorem]{Example}
\newtheorem{exercise}[theorem]{Exercise}
\newtheorem{lemma}[theorem]{Lemma}
\newtheorem{notation}[theorem]{Notation}
\newtheorem{problem}[theorem]{Problem}
\newtheorem{proposition}[theorem]{Proposition}
\newtheorem{remark}[theorem]{Remark}
\newtheorem{solution}[theorem]{Solution}
\newtheorem{summary}[theorem]{Summary}    

\begin{document}

\begin{center}
\large\textbf{Homework 7 \\ COMP543 Fall 2020 - Modern Cryptography \\}
\normalsize\textbf{ Erhan Tezcan 0070881 \\ 20.11.2020} \\
\end{center}

\begin{center}
\line(1,0){250}
\end{center}

%
%\begin{enumerate}[label=\alph*.]
% \item Explain input, output, and the purpose of each algorithm (Key Generation, Encryption, Decryption). 
% \item What are the key space, the message space, and the ciphertext space?
% \item Formally define the   correctness   requirement of an encryption scheme.
% \end{enumerate}
%

%
%\begin{algorithm}
%\caption{\code{DivideRounds} for a new event \code{x}}
%\label{alg:round}
%\begin{algorithmic}
%\STATE $r \gets \text{max}(selfParent.Round, otherParent.Round)$
%\IF {$x \text{ strongly sees } > \frac{2n}{3} \text{ round } r \text{ witnesses}$} 
%        \STATE $x.Round \gets r+1$
%\ELSE
%        \STATE $x.Round \gets r$
%\ENDIF 
%\STATE $x.witness \gets x.Round > x.selfParent.Round$
%\end{algorithmic}
%\end{algorithm}
%
\section{Quesitons}
\textbf{Q1:} Compute the final two (decimal) digits of $3^{1000}$ (by hand). Hint: The answer is $[3^{1000} \bmod 100]$.

\textbf{A1:} Remember a theorem\footnote{Theorem 8.19 from KL Book 2nd ed.} and corollary\footnote{Corollary 8.15 from KL Book 2nd ed.}:
\begin{theorem}
Let $N=\Pi_ip_i^{e_i}$ where $\{p_i\}$ are distinct primes and $e_i \geq 1$. Then we can find the order:
$$
|\mathbb{Z}_N^*|=\phi(N)=\Pi_ip_i^{e_i-1}(p_i-1)
$$
\end{theorem}
\begin{corollary}
For a finite group $G$ with order $m>1$, $\forall x \in G$ and integers $x$ it holds that:
$$
g^x = g^{[x \bmod m]}
$$
\end{corollary}
We have $100 = 2^25^2$ so we can find the order $\mathbb{Z}_100^*= 2^{2-1}(2-1)5^{2-1}(5-1) = 40$. So, $3^{1000} = 3^{[1000 \bmod 40]} \mod 100$. We find the result 1 from this.

\vspace{20px}
\textbf{Q2:} Compute $[4651 \bmod 55]$ (by hand) using the Chinese Remainder Theorem.

\textbf{A2:} We shall notice that $55 = 11 \times 5$ and $gcd(11,5)=1$. CRT says that $\mathbb{Z}_{55} \simeq \mathbb{Z}_{11} \times \mathbb{Z}_{5}$. An instance of this isomorphism is:
$$
4651 \bmod 55 \iff (9 \bmod 11)(1 \bmod 5)
$$
Then we apply the algorithm in page 301 (right above example 8.30) in KL Book 2nd ed.:
\begin{enumerate}
	\item $x\times5 + y\times 11 \implies x = -2, y = 1$
	\item $ 1_p = 11 \mod 55, 1_q = -10 \mod 55 = 45 \mod 55$
	\item $x = (1 \times 11 + 9 \times 45) \mod 55 = 31 \mod 55$
\end{enumerate}
The result is therefore 31.

\vspace{20px}
\textbf{Q3:} What is a group? What are the properties of a group? Which groups are called abelian? Explain your answer.

\textbf{A3:} A group is a way to reason about objects with same underlying nature and share the same mathematical structure. More formally, a group $(G, \cdot)$is a set $G$ along with a binary operation $\cdot$ for which the following hold:
\begin{itemize}
 \item \textbf{(Closure)}: $\forall g, h \in G : g \cdot h \in G$
 \item \textbf{(Existence of Identity)}: $\exists e \in G : \forall g \in G$ s.t. $e \cdot g = g \cdot e = g$
 \item \textbf{(Existence of Inverses)}: $\forall g \in G : \exists h \in G$ s.t. $h \cdot g = g \cdot h = e$. We also denote and inverse of $g$ as $h = g^{-1}$
 \item \textbf{(Associativity)}: $\forall g_1, g_2, g_3 \in G : (g_1 \cdot g_2) \cdot g_3 = g_1 \cdot (g_2 \cdot g_3)$
\end{itemize}
A group $(G, \cdot)$ is Abelian if the binary operation $\cdot$ is \textbf{commutative}: $\forall g, h \in G : g \cdot h = h \cdot g$.

\vspace{20px}
\textbf{Q4:} For each of the following groups, write down their elements and inverses\footnote{The inverse of $x$ in $\mathbb{Z}_N$ is an element y in $\mathbb{Z}_N$ such that $xy = 1$ in $\mathbb{Z}_N$.} of each elements.
\begin{enumerate}[label=\alph*.]
 \item $\mathbb{Z}_6$
 \item $\mathbb{Z}_6^*$
 \item $\mathbb{Z}_7$
 \item $\mathbb{Z}_7^*$
\end{enumerate}

\textbf{A4:}
\begin{enumerate}[label=\alph*.]
 \item $\mathbb{Z}_6 = \{0, 1, 2, 3, 4, 5\}$, only 1 and 5 have an inverse, explained in next bullet.
 \item $\mathbb{Z}_6^* = \{1, 5\}$ and their inverses are respectively: $\{1, 5\}$.
 \item $\mathbb{Z}_7 = \{0, 1, 2, 3, 4, 5, 6\}$, all except 0 have an inverse, explained in the next bullet.
 \item $\mathbb{Z}_7^* = \{1, 2, 3, 4, 5, 6\}$ and their inverses are respectively: $\{1, 4, 5, 2, 3, 6\}$.
\end{enumerate}

\vspace{20px}
\textbf{Q5:} What is a cyclic group? Is the multiplicative group $\mathbb{Z}_7^*$ a cyclic group? If yes, what is the generator? What is the order of element 2 in this group? Explain your answer.

\textbf{A5:} A group $G$ where $|G|=n$ is cyclic if $\exists g \in G$ s.t. $\{1, g^1, g^2, \ldots, g^{n-2}\} = G$, and such $g$ is called a ``generator''. The group $\mathbb{Z}_7^* = \{1, 2, 3, 4, 5, 6\}$ is a cyclic group, which we know from Euler's theorem that says for a prime $p$ the group $\mathbb{Z}_p^*$ is cyclic. The generator of this group is 3: $\{1, 3^1, 3^2, 3^3, 3^4, 3^5\} = \{1, 3, 2, 6, 4, 5\} = \mathbb{Z}_7^*$. The order of 2, also shown as $ord_7(2)$ is 3, because $\{1, 2^1, 2^2, 2^3, 2^4, 2^5\}=\{1, 2, 4\}$ and the size of this group is 3.


\vspace{20px}
\textbf{Q6:} Formally define the RSA assumption.

\textbf{A6:} The RSA assumption is that there exists a \code{GenRSA} algorithm relative to which the RSA problem is \textbf{hard}. \code{GenRSA} is a probabilistic polynomial-time algorithm that when given $1^n$, it outputs a modulus $N$ that is the product of two $n$-bit long primes $e$ and $d$ such that $gcd(e,\phi(N))=1$ and $ed = 1 \bmod \phi(N)$.

\vspace{20px}
\textbf{Q7:} Formally define the Discrete Logarithm assumption.

\textbf{A7:} The Discrete Logarithm assumption is the assumption that solving the discrete logarithm $log_gh$ for a cyclic group $G$, generator $g$ and $h \in G$, is \textbf{hard}.

\end{document}